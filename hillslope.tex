\documentclass{article}

\usepackage{amsmath}
%\usepackage{amsfonts}
\usepackage{amsthm}
%\usepackage{amssymb}
%\usepackage{mathrsfs}
%\usepackage{fullpage}
%\usepackage{mathptmx}
\usepackage[varg]{txfonts}
\usepackage{color}
\usepackage[charter]{mathdesign}
\usepackage[pdftex]{graphicx}
%\usepackage{float}
%\usepackage{hyperref}
%\usepackage[modulo, displaymath, mathlines]{lineno}
%\usepackage{setspace}
%\usepackage[titletoc,toc,title]{appendix}

%\linenumbers
%\doublespacing

\theoremstyle{definition}
\newtheorem*{defn}{Definition}
\newtheorem*{exm}{Example}

\theoremstyle{plain}
\newtheorem*{thm}{Theorem}
\newtheorem*{lem}{Lemma}
\newtheorem*{prop}{Proposition}
\newtheorem*{cor}{Corollary}

\newcommand{\argmin}{\text{argmin}}
\newcommand{\ud}{\hspace{2pt}\mathrm{d}}
\newcommand{\bs}{\boldsymbol}
\newcommand{\PP}{\mathsf{P}}

\title{Action principles for nonlinear hillslope transport}
\author{Daniel Shapero}
\date{}

\begin{document}

\maketitle

The elevation $z$ of a sedimentary landscape with an uplift rate $U$ evolves according to the nonlinear parabolic PDE
\begin{equation}
    \frac{\partial z}{\partial t} = \nabla\cdot\frac{k\nabla z}{1 - S_c^{-2}|\nabla z|^2} + \frac{\rho_r}{\rho_s}U.
\label{hillslope-eqn}\end{equation}
The right-hand side of this equation can be written as the derivative of an \emph{action functional}.
The action is
\begin{equation}
    J(z) = -\int_\Omega\left\{\frac{k S_c^2}{2}\ln\left(1 - S_c^{-2}|\nabla z|^2\right) + \rho_r/\rho_s U\cdot z\right\}\ud x.
\end{equation}
This means that, for any perturbation $w$ satisfying the boundaring conditions for the original problem,
\begin{equation}
    J(z + \delta\cdot w) = J(z) - \delta\int_\Omega\left\{\frac{k\nabla z\cdot \nabla w}{1 - S_c^{-2}|\nabla z|^2} + \rho_r/\rho_s U\cdot w\right\}\ud x + \mathcal{O}(\delta^2).
\end{equation}
The first-order term in $\delta$ is the weak form of the right-hand side of \eqref{hillslope-eqn}.

One consequence is that hillslope evolution is a \emph{gradient flow}.
This means that
\begin{equation}
    \frac{\partial z}{\partial t} = -dJ(z).
\end{equation}
The nice part about gradient flows is that the action $J$ always decreases along trajectories.
We can see this by calculating the time derivative of $J(z)$:
\begin{equation}
    \frac{\ud}{\ud t}J(z) = dJ(z)\cdot\frac{\partial z}{\partial t} = -\|dJ(z)\|^2 \le 0.
\end{equation}
The properties of $J$ can then tell us about the stability of the dynamical system.
In our case, $J$ is a convex functional on the set of all elevation fields $z$ such that the maximum slope of $z$ is less than the critical slope $S_c$.

This analysis assumes that the uplift rate $U$ is constant in time.
If $U$ is time-dependent, we have to add terms including the explicit dependence of $J$ on time, but the idea is still the same.

From a numerical standpoint, we can leverage the fact that $J$ is convex to come up with optimal nonlinear solvers for \eqref{hillslope-eqn}.
To solve a nonlinear equation, you have to specify a convergence tolerance, but often the tolerance has to be tuned for your specific problem.
This tuning doesn't necessarily carry over to other problems posed with different input data or on different geometries.
If the problem comes from a convex action functional, however, the positive part of the action gives a dimensional scale for the problem that we can use to set a tolerance of the correct units, given a desired non-dimensional tolerance.

Action principles are convenient for other reasons beyond making the numerical analysis easier.
The value of the action is an ``objective'' quantity, regardless of how you're approximating the solution of the problem.
If you're using finite difference methods and I'm using finite element methods, we can both compare the value of the action that we found for the same problem, even though our solutions themselves aren't directly comparable without interpolating one representation to the other.
Second, from a conceptual viewpoint, a lot of the post-processing you might want to do on the solution of a PDE amounts to evaluating some linear or nonlinear functional.
Action principles are nice because the specification of the problem and the analysis of the solution use the same conceptual vocabulary.

\end{document}
