\documentclass{article}

\usepackage{amsmath}
%\usepackage{amsfonts}
\usepackage{amsthm}
%\usepackage{amssymb}
%\usepackage{mathrsfs}
%\usepackage{fullpage}
%\usepackage{mathptmx}
%\usepackage[varg]{txfonts}
\usepackage{color}
\usepackage[charter]{mathdesign}
\usepackage[pdftex]{graphicx}
%\usepackage{float}
%\usepackage{hyperref}
%\usepackage[modulo, displaymath, mathlines]{lineno}
%\usepackage{setspace}
%\usepackage[titletoc,toc,title]{appendix}

%\linenumbers
%\doublespacing

\theoremstyle{definition}
\newtheorem*{defn}{Definition}
\newtheorem*{exm}{Example}

\theoremstyle{plain}
\newtheorem*{thm}{Theorem}
\newtheorem*{lem}{Lemma}
\newtheorem*{prop}{Proposition}
\newtheorem*{cor}{Corollary}

\newcommand{\argmin}{\text{argmin}}
\newcommand{\ud}{\hspace{2pt}\mathrm{d}}
\newcommand{\bs}{\boldsymbol}
\newcommand{\PP}{\mathsf{P}}

\title{Action principles for nonlinear hillslope transport}
\author{Daniel Shapero}
\date{}

\begin{document}

\maketitle

The elevation $z$ of a sedimentary landscape with an uplift rate $U$ evolves according to the nonlinear parabolic PDE
\begin{equation}
    \rho_s\left(\frac{\partial z}{\partial t} - \nabla\cdot\frac{k\nabla z}{1 - S_c^{-2}|\nabla z|^2}\right) = \rho_r U.
\end{equation}
The nonlinear part of this operator can be derived from an action principle.
Define the functional
\begin{equation}
    J(z) = -\frac{k S_c^2}{2}\int_\Omega \ln\left(1 - S_c^{-2}|\nabla z|^2\right)\ud x.
\end{equation}
The nonlinear part of the hillslope evolution operator is the derivative of this functional.
One of the consequences of this fact is that hillslope evolution is a \emph{gradient flow}, in the dynamical systems sense.
This means that
\begin{equation}
    \left\langle\frac{\partial z}{\partial t}, w\right\rangle = -\langle \ud J(z), w\rangle + \frac{\rho_r}{\rho_s}\langle U, w\rangle
\end{equation}
where $\langle\cdot, \cdot\rangle$ is the duality pairing.

\end{document}
